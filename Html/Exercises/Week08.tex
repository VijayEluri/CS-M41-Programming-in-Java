% Oliver Kullmann, 20.11.2014 (Swansea)

\documentclass[11pt]{article}
\usepackage[driverfallback=hypertex]{hyperref}
\usepackage{amsmath}
\usepackage{amsfonts}
\usepackage{amssymb}
\usepackage{fullpage}
\usepackage[scaled]{beramono}
\usepackage[T1]{fontenc}

\usepackage{listings}
\lstloadlanguages{Java}
\newcommand{\Java}{\lstset{language=Java,keywordstyle=\bfseries,breaklines,breakindent=30pt}}
\newcommand{\inl}[1]{\lstinline$#1$}

\newcommand{\chp}{http://cs.swan.ac.uk/~csoliver/ProgrammingJava201415_MgQxuCUrrS/index.html}
\newcommand{\module}{CS-M41 Programming in Java 2014/15}


%%% Local Variables: 
%%% mode: latex
%%% TeX-master: t
%%% End: 


\begin{document}

\begin{center}
  \href{http://www.swan.ac.uk/}{Swansea University}\\
  \href{http://www.swan.ac.uk/compsci/}{Computer Science Department}\\[1ex]
  \href{\chp}{\module}\\[1ex]
  \textbf{Lab classes - Week 8}\\
  \href{http://cs.swan.ac.uk/~csoliver}{Oliver Kullmann} 20+24/11/2017
\end{center}


\section{Standard exercises}
\label{sec:stdex}

\Java

If you didn't finish the exercises from last week, then please finish them now (we are still on functions). Always test your programs thoroughly!

\begin{enumerate}
\item Write a function (static method) \texttt{leq}, taking two \texttt{int}'s $a, b$ and returning \texttt{true} if $a \le b$, and \texttt{false} otherwise.
\item Write a function \texttt{sgn}, taking a \texttt{double} $x$ as argument, and returning a \texttt{double}, which is $-1, 0 , +1$ according to $x < 0$, $x = 0$, or $x > 0$.
\item Write a function \texttt{fill}, which takes three \texttt{int}'s, and returns an integer array of length $3$, containing these three numbers in the given order.
\item Write a function \texttt{sort}, which takes three \texttt{int}'s, and returns an integer array of length $3$, containing these three numbers in ascending sorted order.
\item Write a function \texttt{pow2}, which takes an integer argument $e$, and returns the \texttt{long} $2^e$, asserting that $e$ allows for precise computation.
\item Write a function \texttt{eqarr}, which takes two \texttt{int}-arrays as arguments and returns \texttt{true}, if they are equal, and \texttt{false} otherwise; your function should return a value in \emph{every case} (even if the arrays are \texttt{null}).
\item Write a function \texttt{and}, that takes a boolean array as argument, and returns true iff all elements are true; your function should return a value in \emph{every case} (even if the array is \texttt{null}).
\item Write a function \texttt{xor}, that takes a boolean array as argument, and returns true iff an odd number of elements are true; your function should return a value in \emph{every case}.
\item Write a function \texttt{fld} (``floor logarithm dualis''), that takes a floating point number $x > 0$ as input and returns the largest integer $n$ (possibly negative) such that $2^n \le x$ holds. Of course, as usual, do not use any special Java library function -- so implement the function according to definition, via loops.

  Recall: $2^0 = 1$, and $2^{-n} = 1 / 2^n$. Perhaps best to start with the test whether $x = 1$, $x > 1$ or $x < 1$ holds. For example $\mathrm{fld}(128) = 7$, $\mathrm{fld}(127) = 6$, $\mathrm{fld}(0.5) = -1$, $\mathrm{fld}(0.4) = -2$.
\end{enumerate}


\section{Additional exercises}
\label{sec:addex}

Use the Pdf-version of this sheet, in order to get the html-links working.
\begin{itemize}
\item See \href{\chp#ExercisesWeek08}{Course home page, Week 8}.
\item Study the \emph{Solutions}, as provided in the program collection for the module, Chapter 2, Section 2, under \texttt{Completed}.
\end{itemize}



\end{document}
