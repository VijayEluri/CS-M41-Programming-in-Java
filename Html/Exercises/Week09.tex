% Oliver Kullmann, 27.11.2014 (Swansea)

\documentclass[11pt]{article}
\usepackage[driverfallback=hypertex]{hyperref}
\usepackage{amsmath}
\usepackage{amsfonts}
\usepackage{amssymb}
\usepackage{fullpage}
\usepackage[scaled]{beramono}
\usepackage[T1]{fontenc}

\usepackage{listings}
\lstloadlanguages{Java}
\newcommand{\Java}{\lstset{language=Java,keywordstyle=\bfseries,breaklines,breakindent=30pt}}
\newcommand{\inl}[1]{\lstinline$#1$}

\newcommand{\chp}{http://cs.swan.ac.uk/~csoliver/ProgrammingJava201819_4ptSarewFC/index.html}
\newcommand{\module}{CSCM41 Programming in Java 2018/19}


%%% Local Variables: 
%%% mode: latex
%%% TeX-master: t
%%% End: 


\begin{document}

\begin{center}
  \href{http://www.swan.ac.uk/}{Swansea University}\\
  \href{http://www.swan.ac.uk/compsci/}{Computer Science Department}\\[1ex]
  \href{\chp}{\module}\\[1ex]
  \textbf{Lab classes - Week 9}\\
  \href{http://cs.swan.ac.uk/~csoliver}{Oliver Kullmann} 27/11/2014
\end{center}


\section{Standard exercises}
\label{sec:stdex}

\Java

\begin{enumerate}
\item Do the \href{\chp#PractiseLectureWeek09}{two exercises} from the lecture.
\item Leaf through the files \texttt{StdDraw.java} and \texttt{Picture.java}.
\item Write a program \texttt{ShowColour}, that takes from the command-line three integers from 0 to 255 (representing \href{http://en.wikipedia.org/wiki/RGB_color_model}{red, green and blue}), and then creates and shows a 256-by-256 picture of that colour.
\item Write a program \texttt{ExtractColours}, that takes the name of a picture file as a command-line input, and creates four images, the original, one that contains only the red components, one for green, and one for blue. (Hint: Take \texttt{Greyscale} as role model.)
\item Modify program \texttt{AlbersSquares} to take nine command-line arguments, representing now three colours, and then draw the 3*2=6 squares showing all the \href{http://en.wikipedia.org/wiki/Josef_Albers}{Albers squares} with the large square in each colour and the small square in each different colour.
\end{enumerate}


\section{Additional exercises}
\label{sec:addex}

Use the Pdf-version of this sheet, in order to get the html-links working.
\begin{itemize}
\item See \href{\chp#ExercisesWeek09}{Course home page, Week 9}.
\item Some \emph{Solutions} are provided in the program collection for the module, Chapter 3, Section 1, under \texttt{Completed}.
\end{itemize}



\end{document}
