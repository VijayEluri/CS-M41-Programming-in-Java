% Oliver Kullmann, 4.11.2014 (Swansea)

\documentclass[11pt]{article}
\usepackage[driverfallback=hypertex]{hyperref}
\usepackage{amsmath}
\usepackage{amsfonts}
\usepackage{amssymb}
\usepackage{fullpage}
\usepackage[scaled]{beramono}
\usepackage[T1]{fontenc}

\usepackage{listings}
\lstloadlanguages{Java}
\newcommand{\Java}{\lstset{language=Java,keywordstyle=\bfseries,breaklines,breakindent=30pt}}
\newcommand{\inl}[1]{\lstinline$#1$}

\newcommand{\chp}{http://cs.swan.ac.uk/~csoliver/ProgrammingJava201819_4ptSarewFC/index.html}
\newcommand{\module}{CSCM41 Programming in Java 2018/19}


%%% Local Variables: 
%%% mode: latex
%%% TeX-master: t
%%% End: 


\begin{document}

\begin{center}
  \href{http://www.swan.ac.uk/}{Swansea University}\\
  \href{http://www.swan.ac.uk/compsci/}{Computer Science Department}\\[1ex]
  \href{\chp}{\module}\\[1ex]
  \textbf{Lab classes - Week 6}\\
  \href{http://cs.swan.ac.uk/~csoliver}{Oliver Kullmann} 7+9/11/2018
\end{center}


To complete basic programming (Chapter 1 of the book), today some more exercises --- perhaps you want to do some at home.


\section{Standard exercises}
\label{sec:stdex}

\Java

Always test your programs thoroughly!

\begin{enumerate}
\item If you feel insecure, then perhaps you might start with exercises from previous weeks (see the \href{\chp}{Course home page}). Make sure you understand
  \begin{enumerate}
  \item Primitive types like \inl{int} or \inl{boolean}.
  \item The basics of the computation of expressions (including conversions).
  \item Variable declarations and initialisations.
  \item Handling of the command-line.
  \item Simple straight-line programs (just assignments and input/output).
  \item If-then-else.
  \item For- and while-loops.
  \item Arrays.
  \end{enumerate}
\item Write a program \texttt{MinMax} that reads in integers (as many as the user enters) from standard input and prints out the maximum and minimum values. Before entering the program, make a sketch on paper.
\item Write a program \texttt{Create} which reads the integer $N$ from the command-line and prints out, separated by spaces, the numbers $1, \dots, N$. Use this as input for \texttt{MinMax}. Can you run this chain with $N=10^{10}$ ?!:
\begin{verbatim}
> java Create 10000000000 | java MinMax
1
10000000000
\end{verbatim}
Here ``>'' stands for the command-line prompt. Timing on my rusty laptop:
\begin{verbatim}
> time java Create 1000000 | java MinMax
1
1000000
real    0m1.687s
user    0m3.569s
sys     0m0.437s
\end{verbatim}
So $10^6$ needs roughly 1.5s, and thus $10^{10}$ would need roughly $15000s$, say 4 hours. The lab-machines should be faster; you could run the computation in a separate window.
\item Modify \texttt{MinMax} by rejecting non-positive entries, asking the user to re-enter the input (until he gets it correct). Before entering the program, extend the previous sketch. Hint: a while-loop is appropriate for testing the input values.
\item Read $N$ from the command-line and $N$ floating-point numbers from standard input, and compute their mean value. Create an error message if not enough numbers were entered.
\end{enumerate}


\section{A bit more advanced}
\label{sec:advanced}

\begin{enumerate}
\item Read $N$ from the command-line plus $N-1$ integers $1 \le x \le N$ from standard input (or less if not available), ignoring repeated numbers or numbers outside of the range. Then determine the first missing number (from the interval $1 \dots N$):
\begin{verbatim}
> java Missing 4
0 0 5 5 1 1 4 4 4
2
> java Missing 4
4 3 2 1
1
> java Missing 4
2 3
1
\end{verbatim}
\item In the above mean-computation, compute also the standard deviation as the square root of the sum of the squares of the differences from these numbers and the average, the whole then divided by $N-1$. Hint: For this computation you could store the numbers in an array. Actually, with some mathematical knowledge you can avoid storing these numbers, but for the exercise it's okay.
\end{enumerate}


\section{Additional exercises}
\label{sec:addex}

See \href{\chp#ExercisesWeek06}{Course home page, Week 6}.


\end{document}
