% Oliver Kullmann, 4.12.2014 (Swansea)

\documentclass[11pt]{article}
\usepackage[driverfallback=hypertex]{hyperref}
\usepackage{amsmath}
\usepackage{amsfonts}
\usepackage{amssymb}
\usepackage{fullpage}
\usepackage[scaled]{beramono}
\usepackage[T1]{fontenc}

\usepackage{listings}
\lstloadlanguages{Java}
\newcommand{\Java}{\lstset{language=Java,keywordstyle=\bfseries,breaklines,breakindent=30pt}}
\newcommand{\inl}[1]{\lstinline$#1$}

\newcommand{\chp}{http://cs.swan.ac.uk/~csoliver/ProgrammingJava201819_4ptSarewFC/index.html}
\newcommand{\module}{CSCM41 Programming in Java 2018/19}


%%% Local Variables: 
%%% mode: latex
%%% TeX-master: t
%%% End: 


\begin{document}

\begin{center}
  \href{http://www.swan.ac.uk/}{Swansea University}\\
  \href{http://www.swan.ac.uk/compsci/}{Computer Science Department}\\[1ex]
  \href{\chp}{\module}\\[1ex]
  \textbf{Lab classes - Week 10}\\
  \href{http://cs.swan.ac.uk/~csoliver}{Oliver Kullmann} 5/12/2014
\end{center}


\section{Standard exercises}
\label{sec:stdex}

\Java

\begin{enumerate}
\item Write a class \texttt{Trivial}, which stores two integers, and as member functions (``methods'') allows to compute their sum and their product, and allows equality-comparison and translation into strings.
\item Write then a program which reads four integers from the command-line, creates two objects of type Trivial accordingly, and outputs their sums and products, and the objects themselves.
\end{enumerate}


\section{Additional exercises}
\label{sec:addex}

Use the Pdf-version of this sheet, in order to get the html-links working.
\begin{itemize}
\item See \href{\chp#ExercisesWeek10}{Course home page, Week 10}.
\item Some \emph{Solutions} are provided in the program collection for the module, Chapter 3, under \texttt{Week 10}.
\end{itemize}



\end{document}
