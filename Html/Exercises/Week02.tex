% Oliver Kullmann, 6.10.2015 (Swansea)

\documentclass[11pt]{article}
\usepackage[driverfallback=hypertex]{hyperref}
\usepackage{amsmath}
\usepackage{amsfonts}
\usepackage{amssymb}
\usepackage{fullpage}
\usepackage[scaled]{beramono}
\usepackage[T1]{fontenc}

\usepackage{listings}
\lstloadlanguages{Java}
\newcommand{\Java}{\lstset{language=Java,keywordstyle=\bfseries,breaklines,breakindent=30pt}}
\newcommand{\inl}[1]{\lstinline$#1$}

\newcommand{\chp}{http://cs.swan.ac.uk/~csoliver/ProgrammingJava201415_MgQxuCUrrS/index.html}
\newcommand{\module}{CS-M41 Programming in Java 2014/15}


%%% Local Variables: 
%%% mode: latex
%%% TeX-master: t
%%% End: 


\begin{document}

\begin{center}
  \href{http://www.swan.ac.uk/}{Swansea University}\\
  \href{http://www.swan.ac.uk/compsci/}{Computer Science Department}\\[1ex]
  \href{\chp}{\module}\\[1ex]
  \textbf{Lab classes - Week 2}\\
  \href{http://cs.swan.ac.uk/~csoliver}{Oliver Kullmann} 13/10/2016
\end{center}

Below you find embedded links to the course home page, which explicitly is \url{\chp}

\section{Standard exercises}
\label{sec:stdex}

\Java

\begin{enumerate}
\item Try to enter the programs \texttt{Empty}, \texttt{HelloWorld}, \texttt{UseArgument} from the \href{\chp\#LecturesWeek02}{lecture} --- best from memory!
  \begin{center}
    Try, and only then have a glance, try again, glance again, ...
  \end{center}
\item Now write a program \texttt{UseThree}, which is called with
  \begin{lstlisting}{Use}
> java UseThree X Y Z
  \end{lstlisting}
   and which outputs
   \begin{lstlisting}{Use}
     Hello, X, Y, Z!
   \end{lstlisting}
   You find the solution also under \href{\chp\#LecturesWeek02}{material for Lecture 02} --- but try hard!
 \item Write \texttt{Hello10}, which prints out ``Hello, World!'' ten times. Perhaps you want to combine it with using names from the command-line.
\end{enumerate}


\section{Additional exercises}
\label{sec:addex}

See \href{\chp\#ExercisesWeek02}{Course home page, Week 2}.


\end{document}
