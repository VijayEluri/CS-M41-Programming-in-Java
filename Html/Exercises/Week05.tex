% Oliver Kullmann, 30.10.2014 (Swansea)

\documentclass[11pt]{article}
\usepackage[driverfallback=hypertex]{hyperref}
\usepackage{amsmath}
\usepackage{amsfonts}
\usepackage{amssymb}
\usepackage{fullpage}
\usepackage[scaled]{beramono}
\usepackage[T1]{fontenc}

\usepackage{listings}
\lstloadlanguages{Java}
\newcommand{\Java}{\lstset{language=Java,keywordstyle=\bfseries,breaklines,breakindent=30pt}}
\newcommand{\inl}[1]{\lstinline$#1$}

\newcommand{\chp}{http://cs.swan.ac.uk/~csoliver/ProgrammingJava201415_MgQxuCUrrS/index.html}
\newcommand{\module}{CS-M41 Programming in Java 2014/15}


%%% Local Variables: 
%%% mode: latex
%%% TeX-master: t
%%% End: 


\begin{document}

\begin{center}
  \href{http://www.swan.ac.uk/}{Swansea University}\\
  \href{http://www.swan.ac.uk/compsci/}{Computer Science Department}\\[1ex]
  \href{\chp}{\module}\\[1ex]
  \textbf{Lab classes - Week 5}\\
  \href{http://cs.swan.ac.uk/~csoliver}{Oliver Kullmann} 28/10/2015
\end{center}

\section{Standard exercises}
\label{sec:stdex}

\Java

\begin{enumerate}
\item Write a program that declares and initializes an array \texttt{a[]} of size 1000 and accesses \texttt{a[1000]}. Does your program compile? What happens when you run it?
\item Describe and explain what happens when you compile and run a program with the following statement:
  \begin{lstlisting}{large}
final int N = 1000;
final int[] a = new int[N*N*N*N];
  \end{lstlisting}
  Then modify the program to use $N = 200$, and see/understand what happens then.
\item Write a program which reads integers from the command-line (arbitrarily many), stores these numbers in an \texttt{int}-array, and outputs \texttt{true} if all these numbers are different, and \texttt{false} otherwise.
\item Write a program \texttt{Deck32} (recall \texttt{Deck} from the \href{http://cs.swan.ac.uk/~csoliver/ProgrammingJava201415_MgQxuCUrrS/index.html#LecturesWeek05}{lecture}), which uses only a deck of 32 cards, and where the ranks do not include 2 up to 6.
\item Combine \texttt{Deck} and \texttt{Deck32} into one program, where via a command-line argument it is decided whether 32 or 52 cards are to be used.
\end{enumerate}


\section{Additional exercises}
\label{sec:addex}

See \href{\chp#ExercisesWeek05}{Course home page, Week 5}.


\end{document}
