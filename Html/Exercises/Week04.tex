% Oliver Kullmann, 20.10.2015 (Swansea)

\documentclass[11pt]{article}
\usepackage[driverfallback=hypertex]{hyperref}
\usepackage{amsmath}
\usepackage{amsfonts}
\usepackage{amssymb}
\usepackage{fullpage}
\usepackage[scaled]{beramono}
\usepackage[T1]{fontenc}

\usepackage{listings}
\lstloadlanguages{Java}
\newcommand{\Java}{\lstset{language=Java,keywordstyle=\bfseries,breaklines,breakindent=30pt}}
\newcommand{\inl}[1]{\lstinline$#1$}

\newcommand{\chp}{http://cs.swan.ac.uk/~csoliver/ProgrammingJava201415_MgQxuCUrrS/index.html}
\newcommand{\module}{CS-M41 Programming in Java 2014/15}


%%% Local Variables: 
%%% mode: latex
%%% TeX-master: t
%%% End: 


\begin{document}

\begin{center}
  \href{http://www.swan.ac.uk/}{Swansea University}\\
  \href{http://www.swan.ac.uk/compsci/}{Computer Science Department}\\[1ex]
  \href{\chp}{\module}\\[1ex]
  \textbf{Lab classes - Week 4}\\
  \href{http://cs.swan.ac.uk/~csoliver}{Oliver Kullmann} 21/10/2015
\end{center}


\section{Standard exercises}
\label{sec:stdex}

\Java

\begin{enumerate}
\item Write a program which reads two integers and prints them out sorted in ascending order.
\item Write another program which reads three integers and prints them out sorted in ascending order.
\item Write better versions of ``TenHellos.java'' (printing ``Hello World!'' ten times), using first a while-loop and then a for-loop. Which is better?
\item Improve this program to ``NHellos.java'', printing $N \ge 0$ times ``Hello World!'', where $N$ is read from the command-line.

  Extended this program, by reading a second parameter $M \ge 0$ from the command-line, and printing in ``Hello World!'' not just one, but $M$ exclamation marks. For example:
  \begin{lstlisting}{NJ}
> java NHellos 0 7
> java NHellos 1 7
Hello, World!!!!!!!
> java NHellos 1 0
Hello, World
> java NHellos 2 0
Hello, World
Hello, World
> java NHellos 2 3
Hello, World!!!
Hello, World!!!
  \end{lstlisting}

\end{enumerate}


\section{Additional exercises}
\label{sec:addex}

The course home page is \url{http://cs.swan.ac.uk/~csoliver/ProgrammingJava201516_KoU7Z9bzTn/index.html}.

See \href{\chp#ExercisesWeek04}{Course home page, Week 4} for further exercises, and for solutions.


\end{document}
