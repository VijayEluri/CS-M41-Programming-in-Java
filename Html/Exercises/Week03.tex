% Oliver Kullmann, 13.10.2015 (Swansea)

\documentclass[11pt]{article}
\usepackage[driverfallback=hypertex]{hyperref}
\usepackage{amsmath}
\usepackage{amsfonts}
\usepackage{amssymb}
\usepackage{fullpage}
\usepackage[scaled]{beramono}
\usepackage[T1]{fontenc}

\usepackage{listings}
\lstloadlanguages{Java}
\newcommand{\Java}{\lstset{language=Java,keywordstyle=\bfseries,breaklines,breakindent=30pt}}
\newcommand{\inl}[1]{\lstinline$#1$}

\newcommand{\chp}{http://cs.swan.ac.uk/~csoliver/ProgrammingJava201415_MgQxuCUrrS/index.html}
\newcommand{\module}{CS-M41 Programming in Java 2014/15}


%%% Local Variables: 
%%% mode: latex
%%% TeX-master: t
%%% End: 


\begin{document}

\begin{center}
  \href{http://www.swan.ac.uk/}{Swansea University}\\
  \href{http://www.swan.ac.uk/compsci/}{Computer Science Department}\\[1ex]
  \href{\chp}{\module}\\[1ex]
  \textbf{Lab classes - Week 3}\\
  \href{http://cs.swan.ac.uk/~csoliver}{Oliver Kullmann} 19/10/2018
\end{center}


\section{Standard exercises}
\label{sec:stdex}

\Java

\begin{enumerate}
\item Why does 10/3 give 3 and not 3.33333333? And how do we get the latter? (Write a little test program.)
\item What does each of the following print?
  \begin{lstlisting}{Examp}
    System.out.println(2 + "bc");
    System.out.println(2 + 3 + "bc");
    System.out.println((2+3) + "bc");
    System.out.println("bc" + (2+3));
    System.out.println("bc" + 2 + 3);
  \end{lstlisting}
  Explain each outcome, and then write a test program to see whether that's really true.
\item Suppose that a variable a is declared as
  \begin{lstlisting}{Decl}
    int a = 2147483647;
  \end{lstlisting}
  (or equivalently, \texttt{Integer.MAX\_VALUE}). What does each of the following print?
\begin{lstlisting}{ExampI}
  System.out.println(a);
  System.out.println(a + 1);
  System.out.println(2 - a);
  System.out.println(-2 - a);
  System.out.println(2 * a);
  System.out.println(4 * a);
  \end{lstlisting}
Explain each outcome, and then write a test program to see whether that's really true.
\item What is the value of
  \begin{lstlisting}{Num}
    (Math.sqrt(2) * Math.sqrt(2) == 2)
  \end{lstlisting}
  Check it out!
\item Write a program that takes two positive integers as command-line arguments, and prints \texttt{true} iff either divides the other. So for example for inputs $2,4$ or $9,3$ we get \texttt{true}, while for inputs $5,7$ we get \texttt{false}.
\end{enumerate}


\section{A bit more advanced}
\label{sec:advanced}

\begin{enumerate}
\item Start with writing a program yourself, which takes an \texttt{int} $N$ (precisely that!) as an argument, and computes the sum $1 + \dots + N$.

  Consider the usage of this program as ``internal'', that is, we do not handle missing or wrong command-line arguments (but we want to concentrate on the job of proper summation).
\item What's wrong with the following program?:
  \begin{lstlisting}{Sum0}
class Sum0 {
  public static void main(final String[] args) {
    final int N = Integer.parseInt(args[0]);
    int sum = 0;
    for (int i = 1; i <= N; ++i) sum += i;
    System.out.println(sum);
  }
}
  \end{lstlisting}
\item Alright, then let's try
  \begin{lstlisting}{Sum1}
class Sum1 {
  public static void main(final String[] args) {
    final int N = Integer.parseInt(args[0]);
    long sum = 0;
    for (int i = 1; i <= N; ++i) sum += i;
    System.out.println(sum);
  }
}
  \end{lstlisting}
  That does now the job, or?!? (By the way, remember, $10^9$ is a nice number in this context; time the run-time to get a feeling for speed.)
\item Think harder! There is precisely one input for $N$ where the program doesn't work. By the way, understand the output for $N \le 0$ (which is always correct here).
\item Write now a correct program, where the loop-variable \texttt{i} still is an \texttt{int}.
\end{enumerate}

The programs \texttt{Sum0-3.java} you find in \url{\chpbase/CS-M41_Programs/Chapter1/Section2}.


\section{Additional exercises}
\label{sec:addex}

The course home page is \url{\chp}. See \href{\chp#ExercisesWeek03}{Course home page, Week 3} for further exercises.


\end{document}
